\section{Introduction}
Recent research indicate that datacenters will be responsible for between 3\% and 5\% of total energy consumption worldwide by 2030 \cite{AndraeE15}. With the disastrous impact climate change could have, there are environmental as well as business imperatives to improve the efficiency of datacenters. From 2010 onwards, datacenters became more energy efficient by reducing energy spent on `overhead' \cite{AvgerinouBC17} i.e. energy consumed by fans, pumps, transformers and other auxillary equipment. Despite these efforts the total energy consumed by datacenters doubled from 2010-2020 \cite{DoddAGC20} and there are diminishing returns to reducing overhead further. Recent work has identified that the computing components of the datacenter, e.g. servers and switches, are now where the greatest efficiency improvements can be made \cite{DoddAGC20}.

Network function virtualization is a technology that will allow datacenter components to be used more efficiently. A network function is a datacenter component that performs a specific task such as load balancing or packet inspection. Services, such as phone call handling or video streaming, are formed by directing traffic through network functions in a prescribed order. Traditionally, these functions were provided by \lq middleboxes\rq\ implemented in purpose-built hardware. However, middleboxes cannot easily be reconfigured, added or removed from the datacenter and hence consume energy even when not required. Virtual network functions (VNFs) provide the same functionality as middleboxes but run on software that is executed on a virtual machine. New instances of VNFs can be created or destroyed in seconds \cite{AbritaSAA18} allowing the datacenter to spend energy on services only when necessary.

To use VNFs efficiently, the optimal number of instances of each VNF must be placed in the datacenter. In the VNF Placement Problem (VNFPP) there are multiple conflicting quality of service (QoS) metrics (e.g. expected latency, packet loss) for each service in the network, which must also be balanced against the energy consumption of the datacenter. A VNFPP instance defines a set of services to provide and a datacenter topology. A solution to the problem defines where VNFs for each service should be placed and how packets should traverse the datacenter to form services.

Many means of solving the VNFPP have been explored however no algorithm has had widespread acceptance. Whilst exact \cite{BariCAB15,BaumgartnerRB15,MiottoLCG19}, heuristic \cite{GuoWLQA0Y20,QiSW19,QuASK17} and metaheuristic \cite{BillingsleyLMMG19,RankothgeMLRL15,SoualahMGZ17} optimization methods have been considered, existing algorithms are limited by the speed they can evaluate solutions to large problems. Packet level models of a datacenter such as discrete event simulators, accurately measure the datacenter metrics \cite{Pongor93} but are slow to converge. Surrogate models are used in many works \cite{AlameddineQA17,GuoWLQA0Y20,KuoLLT18,QiSW19,QuASK17,RankothgeLRL17,VizarretaCMMK17} as indicative measures of the solution quality. However, it has never been shown that optimizing a surrogate is analogous to optimizing the datacenter metrics. The fastest, accurate models use queueing theory. Queueing theory models can closely approximate the datacenter metrics efficiently by modelling the behavior of queues in the datacenter. Despite this, most queueing theory models consider a simplified representation of the datacenter where queues are allowed to be infinite and hence packets are never dropped \cite{AgarwalMCD18,GouarebFA18,OljiraGTB17} whereas in practice, queues are finite and packet loss is significant. In a previous paper we proposed a fast and accurate queueing model that corrected this issue \cite{BillingsleyLMMG20}, however on large problems this model is slow to converge.

One means of remedying this issue is to utilise the capabilities of modern CPUs, in particular support for high degrees of multithreading. Evolutionary algorithms are parallelisable and appropriate for the VNFPP as they can efficiently find approximate solutions to challenging optimisation problems. Parallel evolutionary algorithms can achieve sub or near linear \cite{BrankeSDM04,El-AlfyA16,LuckenBS04,LuoFBP19,ShiZS20,ZhangYCJ20} and even super-linear \cite{Alba02,MuhlenbeinSB91} reductions in execution time for increasing numbers of threads. However, increased parallelisation can also come at the cost of poorer solutions \cite{BrankeSDM04} or increased execution time \cite{El-AlfyA16}.

In this work we determine whether PMOEAs can efficiently and effectively solve the VNFPP. First, in Section \ref{sec:problem_formulation} we formally specify the VNFPP problem. Next, in Section \ref{sec:algorithms} we describe the PMOEAs considered in this work. Section \ref{sec:results} presents the results of our evaluation. Finally, Section \ref{sec:conclusions} summarizes the overall contributions.